%%%%%%%%%%%%%%%%%%%%%%%%%%%%%%%%%%%%%%%%%%%%%%%%%%%%%%%%%%%%%%%%%%%%%%%%%
%                                                                       %
% ustthesis_test.tex: A template file for usage with ustthesis.cls      %
%                                                                       %
%%%%%%%%%%%%%%%%%%%%%%%%%%%%%%%%%%%%%%%%%%%%%%%%%%%%%%%%%%%%%%%%%%%%%%%%%

\documentclass{ustthesis}


\usepackage{times,amsmath,epsfig}
\usepackage{graphicx,algorithm}
\usepackage[noend]{algorithmic}
\usepackage[center]{subfigure}
\usepackage{xcolor,graphicx}
\usepackage{amsmath}
\newtheorem{proof}{Proof}
\usepackage[left=2.5cm,top=4.0cm,bottom=2.5cm, right=2.5cm]{geometry}

\newcommand{\red}[1]{#1}
\newcommand{\tab}[1]{\hspace{3mm}}


\usepackage[backend=biber, style=numeric]{biblatex}
\addbibresource{ref.bib}
\usepackage[]{hyperref}
\hypersetup{
    linkbordercolor=green,
}

% \usepackage{latexsym}
    % Use the "latexsym" package when encountering the following error:
    %   ! LaTeX Error: Command \??? not provided in base LaTeX2e.
% \usepackage{epsf}
    % Use the "epsf" package for including EPS files.

%%%%%%%%%%%%%%%%%%%%%%%%%%%%%%%%%%%%%%%%%%%%%%%%%%%%%%%%%%%%%%%%%%%%%%%%%
%                                                                       %
% Preambles. DO NOT ERASE THEM. Change to suite your particular purpose.%
%                                                                       %
%%%%%%%%%%%%%%%%%%%%%%%%%%%%%%%%%%%%%%%%%%%%%%%%%%%%%%%%%%%%%%%%%%%%%%%%%

\title{Cardea: A Context-Aware and Interactive Visual Privacy Control Framework}  % Title of the thesis.
\author{Rui~Zheng}     % Author of the thesis.
\degree{\MPhil}             % Degree for which the thesis is.
%% or
%\degree{\PhD}              % Degree for which the thesis is.
\subject{Computer Science and Engineering}      % Subject of the Degree.
\department{Computer Science and Engineering}       % Department to which the thesis
                    % is submitted.
\advisor{Assistant Prof.~Pan.~Hui}     % Supervisor.
\depthead{Prof.~Qiang~Yang}    % department head.
\defencedate{2016}{10}{21}      % \defencedate{year}{month}{day}.

% NOTE:
%   According to the sample shown in the guidelines, page number is
%   placed below the bottom margin.  However, if the author prefers
%   the page number to be printed above the bottom margin, please
%   activate the following command.

% \PNumberAboveBottomMargin

\begin{document}

%%%%%%%%%%%%%%%%%%%%%%%%%%%%%%%%%%%%%%%%%%%%%%%%%%%%%%%%%%%%%%%%%%%%%%%%%
%                                                                       %
% Now the actual Thesis. The order of output MUST be followed:          %
%                                                                       %
%    1) TITLEPAGE                                                       %
%                                                                       %
% The \maketitle command generates the Title page as well as the        %
% Signature page.                                                       %
%                                                                       %
%%%%%%%%%%%%%%%%%%%%%%%%%%%%%%%%%%%%%%%%%%%%%%%%%%%%%%%%%%%%%%%%%%%%%%%%%

\maketitle

%%%%%%%%%%%%%%%%%%%%%%%%%%%%%%%%%%%%%%%%%%%%%%%%%%%%%%%%%%%%%%%%%%%%%%%%%
%                                                                       %
%     2) DEDICATION (Optional)                                          %
%                                                                       %
% The \dedication and \enddedication commands are optional. If          %
% specified it generates a page for dedication.                         %
%
%%%%%%%%%%%%%%%%%%%%%%%%%%%%%%%%%%%%%%%%%%%%%%%%%%%%%%%%%%%%%%%%%%%%%%%%%

% \dedication
% This is an optional section.
% \enddedication

%%%%%%%%%%%%%%%%%%%%%%%%%%%%%%%%%%%%%%%%%%%%%%%%%%%%%%%%%%%%%%%%%%%%%%%%%
%                                                                       %
%     3) ACKNOWLEDGMENTS                                                %
%                                                                       %
% \acknowledgments and \endacknowledgments defines the                  %
% Acknowledgments of the author of the Thesis.                          %
%                                                                       %
%%%%%%%%%%%%%%%%%%%%%%%%%%%%%%%%%%%%%%%%%%%%%%%%%%%%%%%%%%%%%%%%%%%%%%%%%

\acknowledgments

First I would like to thank my family for their unconditional support during the past three years. It is their optimistic attitudes as well as patience that helped me walk across all the troubled water among the years.

I would also like to express my deepest gratitude to my advisor Prof. Pan Hui for his support, guidance and encouragement, without which this thesis would not have been possible.

Special thanks to Jiayu Su, it was a great pleasure to collaborate with her in the past year and I learned a lot from the collaboration. Same thanks gives to Dr. Tongfeng Weng for the enjoyable collaboration on random walks. Other than that, I would like to thank Haris Mughees and Hamza Zia, for countless times we happily talked about everything and I was always surprised by their knowledge, passion as well as determination.

Thanks also goes to all the members in Symlab family. I am fortunate to know so many wonderful people and work as college with them. They have helped me a lot in many aspects from daily life to research. Many thanks to Mr. Issac Ma and Mrs. Connie Lau for their administration work.

Last but not least, I would like to thank Prof. Dit-Yan Yeung and Prof. Chi-Keung Tang. It was great fun to take their courses, which guided me to find my interests and thesis topic. I deeply admire the strong sense of responsibility they have on both lecturing as well as supervision of their students. Their research attitudes and hard working will keep motivating me in my future endeavors.

\endacknowledgments

%%%%%%%%%%%%%%%%%%%%%%%%%%%%%%%%%%%%%%%%%%%%%%%%%%%%%%%%%%%%%%%%%%%%%%%%%
%                                                                       %
%     4) TABLE OF CONTENTS                                              %
%                                                                       %
%%%%%%%%%%%%%%%%%%%%%%%%%%%%%%%%%%%%%%%%%%%%%%%%%%%%%%%%%%%%%%%%%%%%%%%%%

\tableofcontents

%%%%%%%%%%%%%%%%%%%%%%%%%%%%%%%%%%%%%%%%%%%%%%%%%%%%%%%%%%%%%%%%%%%%%%%%%
%                                                                       %
%     5) LIST OF FIGURES (If Any)                                       %
%                                                                       %
%%%%%%%%%%%%%%%%%%%%%%%%%%%%%%%%%%%%%%%%%%%%%%%%%%%%%%%%%%%%%%%%%%%%%%%%%

\listoffigures

%%%%%%%%%%%%%%%%%%%%%%%%%%%%%%%%%%%%%%%%%%%%%%%%%%%%%%%%%%%%%%%%%%%%%%%%%
%                                                                       %
%     6) LIST OF TABLES (If Any)
%                                                                       %
%%%%%%%%%%%%%%%%%%%%%%%%%%%%%%%%%%%%%%%%%%%%%%%%%%%%%%%%%%%%%%%%%%%%%%%%%

\listoftables

%%%%%%%%%%%%%%%%%%%%%%%%%%%%%%%%%%%%%%%%%%%%%%%%%%%%%%%%%%%%%%%%%%%%%%%%%
%                                                                       %
%     7) ABSTRACT                                                       %
%                                                                       %
% \abstract and \endabstract are used to define a short Abstract for    %
% the Thesis.                                                           %
%                                                                       %
%%%%%%%%%%%%%%%%%%%%%%%%%%%%%%%%%%%%%%%%%%%%%%%%%%%%%%%%%%%%%%%%%%%%%%%%%

\begin{abstract}

The growing popularity of mobile and wearable devices with built–in cameras, the bright prospect of camera related applications such as augmented reality and life–logging system, the increased ease of taking and sharing photos, along with advances in computer vision techniques, have greatly facilitated people’s lives in many aspects, but inevitably raised people’s concerns about visual privacy at the same time.

Motivated by the finding that people’s privacy concerns are influenced by the context, in this thesis, we propose Cardea, a context--aware and interactive visual privacy control framework that enforces privacy policies according to people’s privacy preferences. The framework provides people with fine–grained visual privacy control using:
\begin{inparaenum}[\itshape i\itshape)]
\item personal privacy profiles, with which people can define their context--dependent privacy preferences;
\item natural visual indicators: face features, for devices to automatically locate individuals who request privacy protection;
\item hand gestures, for people to temporarily update and flexibly inform cameras of their privacy preferences.
\end{inparaenum}
Benefited from recent progresses in face and object recognition, Cardea offers a way for context--dependent privacy control in a natural and flexible manner, which differs from tag and marker based systems. We design and implement the framework consisting of Android client app and cloud control server, with convolutional neural networks as core of the image processing module. Our evaluation results confirm such framework is practical and effective, showing promising future for context--aware visual privacy control on mobile and wearable devices.

\end{abstract}


%%%%%%%%%%%%%%%%%%%%%%%%%%%%%%%%%%%%%%%%%%%%%%%%%%%%%%%%%%%%%%%%%%%%%%%%%
%                                                                       %
%     8) The Actual Contents                                            %
%                                                                       %
% The command \chapters MUST BE USED to ensure that the entire content  %
% of the Thesis is double-spaced (in version 1.0).                      %
%                                                                       %
% However, in version 2.0, \chapters will be automatically added in     %
% the beginning of the first chapter.                                   %
%                                                                       %
%%%%%%%%%%%%%%%%%%%%%%%%%%%%%%%%%%%%%%%%%%%%%%%%%%%%%%%%%%%%%%%%%%%%%%%%%

%%\chapters         % Not necessary with ustthesis.cls (v2.0).

%%%%%%%%%%%%%%%%%%%%%%%%%%%%%%%%%%%%%%%%%%%%%%%%%%%%%%%%%%%%%%%%%%%%%%%%%
%                                                                       %
% Each chapter is defined via the \chapter command. The usual sectional %
% commands of LaTeX are also available.                                 %
%                                                                       %
%%%%%%%%%%%%%%%%%%%%%%%%%%%%%%%%%%%%%%%%%%%%%%%%%%%%%%%%%%%%%%%%%%%%%%%%%

\chapter{Introduction}\label{sec-introduction}

The concern about visual privacy has been growing in last decade with increasing adoption of video surveillance systems for security reasons. The statistics shows there are 125 video surveillance cameras per thousand people in U.S. by 2014~\cite{links:numofsurv}. Momentum of new technologies such as the Internet of Things (IOT) will keep driving global video surveillance market in following years, which will raise more privacy concerns.

Other than closed-circuit television (CCTV) surveillance systems for security reasons, handheld devices such as camera phones are also used extensively for the recording of meaningful life moments. Recently, coming with the explosion of products in augmented reality (e.g., Google Glass), robotics (e.g., iRobot Create platform), and gaming (e.g., Kinect), is more and more cameras being embedded in these platforms for the enhancement of life experiences. The trend of embedding cameras, especially in wearables, will keep growing, an example of which is smart contact lens~\cite{links:eyecontact}. However, the ubiquitous presence of cameras, the ease of taking photos and recording videos, along with ``always on'' and ``non overt act'' features threaten individuals to have private or anonymous social lives, raising people's concerns of visual privacy.

More specifically, photos and videos captured without getting permissions from bystanders, and then uploaded to social networking sites, can be accessed by everyone online, potentially leading to invasion of privacy. Malicious applications on the device may also inadvertently leak captured media data~\cite{links:appleakspriv}.

Benefited from research breakthroughs from deep learning community~\cite{Goodfellow-et-al-2016-Book}, current vision perception systems are advancing fast in their capabilities of understanding image and video contents~\cite{links:awesomedeepvision}. Nowadays, recognition technologies can link images to specific people~\cite{taigman2014deepface,sun2015deepid3,schroff2015facenet}, places~\cite{weyand2016planet}, and general objects~\cite{russakovsky2015imagenet}, making what previously unsearchable now searchable~\cite{acquisti2014face}, thus reveal far more private information than expected.

Both legal and technical measures have been proposed to resolve visual privacy concerns. For instance, Google Glass is banned at places such as banks, hospitals, and bars~\cite{links:glassbanned}. However, prohibition of cameras usage does not resolve the issue fundamentally, instead it may intrude people's rights to capture happy moments. As a result, there are growing needs to design technical solutions to protect individuals' visual privacy in a world with pervasive cameras. Technical solutions that have been proposed so far are still limited, in the way that they are mostly based on static policies, thus users can not flexibly express their individualized privacy preferences based on surrounding contexts when they are captured. Moreover, previous works require users to wear visual markers such as hats~\cite{schiff2009respectful} for the detection of interested persons, or clip tags such as QR codes~\cite{bo2014privacy,roesner2014world} for the fetching of privacy polices. Despite technical feasibilities of these approaches, the extra need of setting up markers/tags and the resulting aesthetically unpleasant appearance will hinder users' willingness to adopt these solutions.

Therefore, the motivation of this thesis is to seek a more natural, user–friendly, flexible, and fine-grained mechanism for people to express, modify, and control their individualized privacy preferences. Under this guideline, we propose Cardea, a context-aware and interactive visual privacy control framework, which lets individuals control their visual privacies through:
\begin{inparaenum}[\itshape i\itshape)]
\item personal privacy profiles, with which people can define their context–dependent privacy preferences;
\item different visual indicators: face features and tags, for devices to automatically locates individuals who request privacy protection;
\item hand gestures, for people to temporarily update and flexibly inform cameras of their privacy preferences.
\end{inparaenum}
When using Cardea, the device will automatically compute context factors, compare them with people’s privacy profiles, and finally enforce privacy policies conforming to people’s privacy preferences. To our knowledge, this is one of the pioneering works that leverages deep learning models, more specifically convolutional neural networks (CNN)~\cite{lecun1998gradient}, to enable visual privacy control in a context-specific and interactive manner.

The rest of the thesis is organized as follows: We first review and discuss related works on visual privacy control in Chapter 2. Following that we introduce convolutional neural networks, the core of Cardea's image processing module, and their applications on related computer vision problems. We then give details about the design, implementation and evaluation of Cardea in Chapter 4. Finally, we share our thoughts on possible future work and conclude the thesis in Chapter 5.

\newpage


%%%%%%%%%%%%%%%%%%%%%%%%%%%%%%%%%%%%%%%%%%%%%%%%%%%%%%%%%%%%%%%%%%%%%%%%%
%                                                                       %
%      9) BIBLIOGRAPHY                                                  %
%                                                                       %
% This example uses bibtex to generate the required Bibliography. Refer %
% to the % the file ustthesis_test.bib for the entries of the           %
% Bibliography. Note that only the cited entries are printed.           %
%                                                                       %
% If BibTeX is not used to typeset the bibliography, replace the        %
% following line with the \begin{thebibliography} and \end{bibliography}%
% commands (the "thebibliography" environment) to process the           %
% Bibliography.                                                         %
%                                                                       %
%%%%%%%%%%%%%%%%%%%%%%%%%%%%%%%%%%%%%%%%%%%%%%%%%%%%%%%%%%%%%%%%%%%%%%%%%

%%%%%%%%%%%%%%%%%%%%%%%%%%%%%%%%%%%%%%%%%%%%%%%%%%%%%%%%%%%%%%%%%%%%%%%%%
%                                                                       %
% The recommended bibliography style is the IEEE bibliography style.    %
% "ustbib" defines the IEEE bibliography standard with the added        %
% ability of sorting the items by name of author.                       %
%                                                                       %
% If you are not using BibTeX to process your Bibliography, comment out %
% the following line.                                                   %
%                                                                       %
%%%%%%%%%%%%%%%%%%%%%%%%%%%%%%%%%%%%%%%%%%%%%%%%%%%%%%%%%%%%%%%%%%%%%%%%%

% if use bibtex
% \bibliographystyle{plain}
% \bibliography{ref}

% if use biber
\printbibliography[heading=bibintoc]

% Please run "bibtex ustthesis_test" before the bibliography can be
% included.

%%%%%%%%%%%%%%%%%%%%%%%%%%%%%%%%%%%%%%%%%%%%%%%%%%%%%%%%%%%%%%%%%%%%%%%%%
%                                                                       %
%     10) APPENDIX (If Any)                                              %
%                                                                       %
% \appendix command marks the beginning of the APPENDIX part of the     %
% Thesis. The usual \chapter command is used for the different chapters %
% of the Appendix.                                                      %
%                                                                       %
%%%%%%%%%%%%%%%%%%%%%%%%%%%%%%%%%%%%%%%%%%%%%%%%%%%%%%%%%%%%%%%%%%%%%%%%%


%%%%%%%%%%%%%%%%%%%%%%%%%%%%%%%%%%%%%%%%%%%%%%%%%%%%%%%%%%%%%%%%%%%%%%%%%
%                                                                       %
%     11) BIOGRAPHY (Optional)                                          %
%                                                                       %
% \biography and \endbiography are used to define the optional          %
% Biography of the author of the Thesis.                                %
%                                                                       %
%%%%%%%%%%%%%%%%%%%%%%%%%%%%%%%%%%%%%%%%%%%%%%%%%%%%%%%%%%%%%%%%%%%%%%%%%

% \biography
% The biography of the student is ALSO optional.
% \endbiography

\end{document}
