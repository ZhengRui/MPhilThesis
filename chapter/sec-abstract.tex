\begin{abstract}

The growing popularity of mobile and wearable devices with built–in cameras, the bright prospect of camera related applications such as augmented reality and life–logging system, the increased ease of taking and sharing photos, along with advances in computer vision techniques, have greatly facilitated people’s lives in many aspects, but inevitably raised people’s concerns about visual privacy at the same time.

Motivated by the finding that people’s privacy concerns are influenced by the context, in this thesis, we propose Cardea, a context--aware and interactive visual privacy control framework that enforces privacy policies according to people’s privacy preferences. The framework provides people with fine–grained visual privacy control using:
\begin{inparaenum}[\itshape i\itshape)]
\item personal privacy profiles, with which people can define their context--dependent privacy preferences;
\item natural visual indicators: face features, for devices to automatically locate individuals who request privacy protection;
\item hand gestures, for people to temporarily update and flexibly inform cameras of their privacy preferences.
\end{inparaenum}
Benefited from recent progresses in face and object recognition, Cardea offers a way for context--dependent privacy control in a natural and flexible manner, which differs from tag and marker based systems. We design and implement the framework consisting of Android client app and cloud control server, with convolutional neural networks as core of the image processing module. Our evaluation results confirm such framework is practical and effective, showing promising future for context--aware visual privacy control on mobile and wearable devices.

\end{abstract}
