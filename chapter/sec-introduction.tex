\chapter{Introduction}\label{sec-introduction}

The concern about visual privacy has been growing in last decade with increasing adoption of video surveillance systems for security reasons. The statistics shows there are 125 video surveillance cameras per thousand people in U.S. by 2014~\cite{links:numofsurv}. Momentum of new technologies such as the Internet of Things (IOT) will keep driving global video surveillance market in following years, which will raise more privacy concerns.

Other than closed-circuit television (CCTV) surveillance systems for security reasons, handheld devices such as camera phones are also used extensively for the recording of meaningful life moments. Recently, coming with the explosion of products in augmented reality (e.g., Google Glass), robotics (e.g., iRobot Create platform), and gaming (e.g., Kinect), is more and more cameras being embedded in these platforms for the enhancement of life experiences. The trend of embedding cameras, especially in wearables, will keep growing, an example of which is smart contact lens~\cite{links:eyecontact}. However, the ubiquitous presence of cameras, the ease of taking photos and recording videos, along with ``always on'' and ``non overt act'' features threaten individuals to have private or anonymous social lives, raising people's concerns of visual privacy.

More specifically, photos and videos captured without getting permissions from bystanders, and then uploaded to social networking sites, can be accessed by everyone online, potentially leading to invasion of privacy. Malicious applications on the device may also inadvertently leak captured media data~\cite{links:appleakspriv}.

Benefited from research breakthroughs from deep learning community~\cite{Goodfellow-et-al-2016-Book}, current vision perception systems are advancing fast in their capabilities of understanding image and video contents~\cite{links:awesomedeepvision}. Nowadays, recognition technologies can link images to specific people~\cite{taigman2014deepface,sun2015deepid3,schroff2015facenet}, places~\cite{weyand2016planet}, and general objects~\cite{russakovsky2015imagenet}, making what previously unsearchable now searchable~\cite{acquisti2014face}, thus reveal far more private information than expected.

Both legal and technical measures have been proposed to resolve visual privacy concerns. For instance, Google Glass is banned at places such as banks, hospitals, and bars~\cite{links:glassbanned}. However, prohibition of cameras usage does not resolve the issue fundamentally, instead it may intrude people's rights to capture happy moments. As a result, there are growing needs to design technical solutions to protect individuals' visual privacy in a world with pervasive cameras. Technical solutions that have been proposed so far are still limited, in the way that they are mostly based on static policies, thus users can not flexibly express their individualized privacy preferences based on surrounding contexts when they are captured. Moreover, previous works require users to wear visual markers such as hats~\cite{schiff2009respectful} for the detection of interested persons, or clip tags such as QR codes~\cite{bo2014privacy,roesner2014world} for the fetching of privacy polices. Despite technical feasibilities of these approaches, the extra need of setting up markers/tags and the resulting aesthetically unpleasant appearance will hinder users' willingness to adopt these solutions.

Therefore, the motivation of this thesis is to seek a more natural, user–friendly, flexible, and fine-grained mechanism for people to express, modify, and control their individualized privacy preferences. Under this guideline, we propose Cardea, a context-aware and interactive visual privacy control framework, which lets individuals control their visual privacies through:
\begin{inparaenum}[\itshape i\itshape)]
\item personal privacy profiles, with which people can define their context–dependent privacy preferences;
\item different visual indicators: face features and tags, for devices to automatically locates individuals who request privacy protection;
\item hand gestures, for people to temporarily update and flexibly inform cameras of their privacy preferences.
\end{inparaenum}
When using Cardea, the device will automatically compute context factors, compare them with people’s privacy profiles, and finally enforce privacy policies conforming to people’s privacy preferences. To our knowledge, this is one of the pioneering works that leverages deep learning models, more specifically convolutional neural networks (CNN)~\cite{lecun1998gradient}, to enable visual privacy control in a context-specific and interactive manner.

The rest of the thesis is organized as follows: We first review and discuss related works on visual privacy control in Chapter 2. Following that we introduce convolutional neural networks, the core of Cardea's image processing module, and their applications on related computer vision problems. We then give details about the design, implementation and evaluation of Cardea in Chapter 4. Finally, we share our thoughts on possible future work and conclude the thesis in Chapter 5.

\newpage
